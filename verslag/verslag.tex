\documentclass[paper=a4, fontsize=11pt]{report}
\usepackage[T1]{fontenc}
\usepackage{fourier}

\usepackage[english]{babel}															% English language/hyphenation
\usepackage[protrusion=true,expansion=true]{microtype}	
\usepackage{amsmath,amsfonts,amsthm} % Math packages
\usepackage[pdftex]{graphicx}	
\usepackage{url}

%%% Equation and float numbering
\numberwithin{equation}{section}		% Equationnumbering: section.eq#
\numberwithin{figure}{section}			% Figurenumbering: section.fig#
\numberwithin{table}{section}				% Tablenumbering: section.tab#


%%% Maketitle metadata
\newcommand{\horrule}[1]{\rule{\linewidth}{#1}} 	% Horizontal rule

\title{
		%\vspace{-1in} 	
		\usefont{OT1}{bch}{b}{n}
		\normalfont \normalsize \textsc{University of Twente} \\ [25pt]
		\horrule{0.5pt} \\[0.4cm]
		\huge Alia Programming Language \\
		\horrule{2pt} \\[0.5cm]
}
\author{
		\normalfont 								\normalsize
        Fedor Beets\\[-3pt]		\normalsize
        Joost van Doorn\\[-3pt]		\normalsize
        \today
}
\date{}


%%% Begin document
\begin{document}
\maketitle
\section*{Introduction}
%The Alia programming language was build for the compiler construction project at the University of Twente.

%Korte beschrijving van de eindopdracht

\chapter{Alia Programming Language}
%De programmeertaal
%Maximaal een A4

%Typer inference
%Expression language

\chapter{Problems and solutions}
%uitleg over de wijze waarop je de problemen die je bent tegengekomen bij het maken van de opdracht hebt opgelost (maximaal twee A4-tjes)

\chapter{Syntax, context-constraints and semantics}


\chapter{Translation rules} %Vertaalregels, hoofdstuk 7 van Watt & Brown

\chapter{Java-code}
%Beknopte bespreking van extra Java-klassen
%Symbol table management
%Type checking
%Code generation
%Error handeling
%Welke informatie in AST nodes worden opgeslagen
\chapter{Tests}

\chapter{Appendices}
%ANTLR LEXER
%ANTLR PARSER
%ANTLR TREEPARSER
%Een testprogramma (invoer/uitvoer)
% 
%%% End document
\end{document}